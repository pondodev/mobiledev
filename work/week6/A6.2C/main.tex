\documentclass{scrartcl}

\usepackage[T1]{fontenc}
\usepackage[utf8]{inputenc}

\title{Mobile Dev 6.2C}
\author{Daniel Coady (102084174)}
\date{23/10/2019}

\usepackage{graphicx}

\begin{document}

\maketitle

\section*{Introduction}
In this short report, we will discuss how the movie rating Android
application was modified to improve on both the performance and the
usability of the application. This will mostly cover how UX design was
incorporated into the application as well as different technical aspects
to keep in mind while developing an Android application.

\section*{Performance Optimisation}
When optimising for performance there are two main things to keep in mind:
\begin{itemize}
    \item CPU time
    \item Memory usage
\end{itemize}
Everything that you do while developing anything is a tradeoff between
space and time. You could either do a calculation one time and store it
for later use, or you could do a calculation each time that you need
that result. One is not necessarily better than the other and each has
their own pros and cons, but it's something to keep in mind while developing
any solution. In this case we had an issue of high memory usage, capping
out at roughly 75MB of RAM being used at any one time. The reason for this
is due to the list view being used in the main activity. A list view by
design will create all of the fragments it contains and keep them loaded
into memory the entire time, which as you could imagine with a larger
list can create some issues. The way to avoid this in this situation was
to use a recycler view. Recyler views are similar to list views, with
the main difference being that it destroys any fragments that are not in
view currently, saving memory. Changing the list view to a recycler view
cut down our memory usage by roughly 25-30MB which is a sizable improvement.

\pagebreak

\section*{Usability}
A core issue present in this application is the pause when everything is
being loaded in. While this is happening the user is unaware of what
the application is doing which makes for a poor user experience. To
assist with this, while the file is being read from the application
storage we display a progress dialog that informs the user that there
is data loading occuring in the background. This means that the user now
knows what it happening, giving them peace of mind that the application
hasn't become unresponsive.

\begin{figure}[h]
    \begin{verbatim}
    internal inner class loadMovies : AsyncTask<Void, Void, ArrayList<Movie>>() {

        lateinit var loadingDialog: ProgressDialog

        override fun onPreExecute() {
            super.onPreExecute()
            loadingDialog = ProgressDialog.show(this@MainActivity, "", "Loading...")
        }

        override fun doInBackground(vararg p0: Void?): ArrayList<Movie> {
            Thread.sleep(3000)
            return Movie.loadFromFile(resources.openRawResource(R.raw.ratings))
        }

        override fun onPostExecute(result: ArrayList<Movie>?) {
            super.onPostExecute(result)
            loadingDialog.dismiss()

            if (result != null)
                initializeUI(result)
            else
                initializeUI(ArrayList())
        }
    }
    \end{verbatim}
    \caption{The async task class used to load the data in the background}
\end{figure}

\end{document}
